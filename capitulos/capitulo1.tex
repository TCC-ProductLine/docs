\chapter[Introdução]{Introdução}

Este capítulo irá apresentar o tema proposto deste trabalho, o problema que se tentará resolver, os objetivos a serem alcançados e a contextualização do mesmo.
% Falar sobre o crescimento de redes sociais, e ferramentas de apoio à desenvolvimento de softwares

\section{Contextualização}

Uma rede social virtual é uma comunidade online que representa um conjunto de participantes (pessoas, organizações ou outras entidades), unindo ideias e recursos em torno de relacionamentos, valores e interesses compartilhados \cite{Marteleto:2001}.

Diversos sistemas conhecidos parecem estruturados como uma rede. Para a Biologia, há o interesse em saber quem se alimenta de quem, quando se estuda a cadeia alimentar. O cérebro faz ligações entre neurônios; as sinapses, para que a pessoa lembre ou resolva algum problema ou questão. A internet é uma rede na qual as pessoas se conectam e se comunicam. As doenças podem se propagar de uma pessoa para outras, deflagrando uma epidemia \cite{Goular:2014}.

Assim, a análise da relação entre os nós e da estrutura formada pela rede fornece informações a respeito de diversos fenômenos e situações: como o cérebro funciona, como a doença se propaga, como as pessoas se comunicam e trocam informações, isto é, as relações ou interações influenciam a própria rede \cite{Goular:2014}.

A relação entre os nós da rede tem várias denominações apresentadas em trabalhos científicos: vínculo, ligação, arco, interação, conexão, relação. Os nós da rede, também chamados de atores, estão ligados por essas relações. Por exemplo, os atores podem se classificar como amigos, quem troca informação com quem, quem confia em quem. A relação pode indicar que os atores fazem parte de um clube, de uma associação, trabalham no mesmo departamento ou que mantêm transações comerciais, trocam mensagens, trabalham em equipe, cooperam entre si para algum tipo de trabalho. Os atores podem ser pessoas ou empresas que estão relacionadas por alguma atividade, além de grupos, localidades, cidades, regiões, entre outros \cite{Hanneman:Riddle:2005}.

Jhon Scott, em seu livro \cite{Scott:Carrington:2011} diz que redes sociais são formadas por dois tipos principais de dados: dados de atributos e dados de relacionamento. Diz-se que os dados de atributos são as opiniões e comportamentos dos agentes demonstrados dentro da rede. Portanto, são qualidades e características que pertencem à eles como indivíduos. Esses dados podem ser quantificados e analisados. Os dados de relacionamento dizem respeito aos contatos, laços e conexões. Esses dados não são dados de um agente, e sim, de um conjunto de agentes conectados que formam um sistema de relacionamento. É possível quantificar e analisar esses tipos de dados, podendo encontrar padrões de relacionamentos em grupos.

Reutilização de Software é o processo de criar sistemas de software a partir de um software já existente, ao invés de criar a partir do zero \cite{Krueger:1992}.

A meta da reutilização de software é reciclar o \textit{design}, código e outros componentes de um software e, assim reduzir o custo, o tempo e melhorar a qualidade do produto \cite{Keswani:Joshi:Jatain:2014}.

Mesmo com todos os benefícios propostos, ao lidar com reutilização de software, deve-se planejar e conhecer bem quais são os objetivos esperados a partir dessa prática, como dizem os autores em sua publicação sobre reutilização de software: Para que um programa de reutilização de software confira o retorno apropriado, o mesmo deve ser sistematizado e planejado. Uma organização que implementa reutilização deve identificar os melhores métodos e estratégias para alcançar máxima produtividade \cite{Keswani:Joshi:Jatain:2014}.

\section{Questão de Pesquisa}

Esse trabalho tem como intuito colaborar a seguinte questão: É possível oferecer um framework que auxilie no desenvolvimento de redes sociais disponibilizando recursos gerais de relacionamentos e específicos de definições de trajetos e agenda, proporcionando ao desenvolvedor facilidade ao lidar com preocupações intrínsecas desse contexto?

\section{Justificativa}

As redes sociais ultrapassaram o âmbito acadêmico/científico, conquistando espaço em outras esferas. É possível observar esse movimento em esferas como a Internet, onde encontramos adeptos com objetivos específicos \cite{Tomae:Alcara:Chiara:2005}.

Nas últimas décadas, o trabalho pessoal em redes de conexões passou a ser percebido como um instrumento organizacional, apesar de o envolvimento das pessoas em redes fazer parte da história da humanidade há muito tempo \cite{Tomae:Alcara:Chiara:2005}.

% *****************************************************************************************
% Fechar a justificativa, retornando a questão de pesquisa, e concluindo o tema foco do TCC
% *****************************************************************************************

\section{Objetivos}

Esse trabalho contém objetivos geral e específicos, conforme colocado nos subtópicos a seguir apresentados.

\subsection{Objetivo Geral}

Oferecer um framework para ser utilizado no desenvolvimento de Redes Sociais, o qual disponibiliza recursos gerais de relacionamentos e específicos de definição de trajetos e agenda, procurando auxiliar o desenvolvedor de software a lidar com preocupações intrínsecas desse contexto.

\subsection{Objetivos Específicos}

A partir do objetivo geral pôde-se definir os objetivos específicos, seguem os mesmos:

\begin{itemize}
	\item Definir uma arquitetura com base no uso de grafos, alinhadas às boas práticas da Engenharia de Software, para representar os relacionamento entre as pessoas.
	\item Usar estruturas de dados e algoritmos específicos, implementados de forma otimizada, visando desempenho e facilidades na manutenção evolutiva do software.
	\item Instanciar um produto de software - i.e. uma rede social específica - a partir do framework, no intuito de coletar as primeiras impressões acerca do suporte desenvolvido como tema foco desse trabalho.
	\item Orientar-se por uma metodologia que permita trabalhar a investigação do domínio de redes sociais, a identificação de suporte tecnológico adequado, o desenvolvimento do framework, a instanciação do mesmo, e a coleta das primeiras impressões com base nos resultados obtidos.
\end{itemize}

\section{Organização deste trabalho}

Este documento foi dividido em 6 capítulos que são descritos brevemente à seguir:

\begin{itemize}
	\item \textbf{Introdução:} Capítulo para apresentação inicial das idéias propostas para esse trabalho, contém a contextualização, problema, justificativa e os objetivos.
	\item \textbf{Referencial Teórico:} Levantamento bibliográfico de todos os assuntos pertinentes para alcançar o embasamento teórico necessário para a realização deste trabalho.
	\item \textbf{Metodologia:} Apresentação de como será conduzido o trabalho, mostrando métodos de desenvolvimento e métodos de pesquisa utilizados.
	\item \textbf{Suporte Tecnológico:} Exposição de todas as ferramentas e tecnologias que serão utilizadas no decorrer deste trabalho.
	\item \textbf{Proposta:} Apresentação completa do que este trabalho se propõe a fazer.
	\item \textbf{Considerações Finais:} Capítulo destinado aos resultados encontrados até o presente momento e como se dará a continuidade do desenvolvimento proposto.
\end{itemize}
