\chapter[Introdução]{Introdução}

Este capítulo irá apresentar a contextualização deste trabalho, o tema proposto, o problema que se tentará resolver, os objetivos a serem alcançados e como este está organizado.

Desde a sua concepção, sites de redes sociais têm atraído milhões de usuários, muitos dos quais tem integrado esses sites em suas práticas diárias. Existem centenas de redes sociais com diversas capacidades tecnológicas suportando uma ampla gama de interesses e práticas. Os principais recursos tecnológicos envolvidos são bem consistentes e semelhantes, porém, as culturas que estão impregnadas podem variar bastante. Alguns sites tendem a atender diversos tipos de pessoas com interesses variados, outros se voltam a um grupo mais específico que se comunicam em uma linguagem comum ou outros interesses \cite{Boyd:Ellison:2007}.

Nos últimos tempos, sistemas que promovem a interação de pessoas, o compartilhamento de informações e a formação de grupos como as redes sociais, deixaram de ser uma tendência, e se estabeleceram de maneira irreversível. Tal fato tem atraído cada vez mais pessoas para o uso dessas tecnologias, mudando a forma de interação e comunicação entre os indivíduos \cite{Santana:Melo-Solarte:Neris:Miranda:Baranauskas:2009}.

Na América Latina, de acordo com uma publicação de Lipsman \cite{Lipsman:2008} sobre o crescimento de redes sociais, obteve-se um aumento de acessos em 33\%, resultado obtido em uma comparação entre 2007 e 2008.

Visando colaborar com o contexto de redes sociais, busca-se apresentar uma ferramenta que auxilie desenvolvedores a criar sistemas desse tipo, oferecendo recursos que são comuns a maioria das aplicações sociais. Este trabalho propõe-se a desenvolver um \textit{framework} que ofereça esses recursos para o desenvolvedor. Nas próximas seções, essa ideia será contextualizada e justificada.

\section{Contextualização}

Uma rede social virtual é uma comunidade \textit{online} que representa um conjunto de participantes (pessoas, organizações ou outras entidades), unindo ideias e recursos em torno de relacionamentos, valores e interesses compartilhados \cite{Marteleto:2001}.

Diversos sistemas conhecidos parecem estruturados como uma rede. Para a Biologia, há o interesse em saber quem se alimenta de quem, quando se estuda a cadeia alimentar. O cérebro faz ligações entre neurônios; as sinapses, para que a pessoa lembre ou resolva algum problema ou questão. A internet é uma rede na qual as pessoas se conectam e se comunicam. As doenças, por exemplo, podem se propagar de uma pessoa para outras, deflagrando uma epidemia \cite{Goular:2014}.

Assim, a análise da relação entre os nós e da estrutura formada pela rede fornece informações a respeito de diversos fenômenos e situações: como o cérebro funciona, como a doença se propaga, como as pessoas se comunicam e trocam informações, isto é, as relações ou interações influenciam a própria rede \cite{Goular:2014}.

A relação entre os nós da rede tem várias denominações apresentadas em trabalhos científicos: vínculo, ligação, arco, interação, conexão e/ou relação. Os nós da rede, também chamados de atores, estão ligados por essas relações. Por exemplo, os atores podem se classificar como amigos, quem troca informação com quem, quem confia em quem. A relação pode indicar que os atores fazem parte de um clube, de uma associação, trabalham no mesmo departamento ou que mantêm transações comerciais, trocam mensagens, trabalham em equipe ou cooperam entre si para algum tipo de trabalho. Os atores podem ser pessoas ou empresas que estão relacionadas por alguma atividade, além de grupos, localidades, cidades, regiões, entre outros \cite{Hanneman:Riddle:2005}.

John Scott, em seu livro \cite{Scott:Carrington:2011}, diz que redes sociais são formadas por dois tipos principais de dados: dados de atributos e dados de relacionamentos. Diz-se que os dados de atributos são as opiniões e os comportamentos dos agentes demonstrados dentro da rede. Portanto, são qualidades e características que pertencem a eles como indivíduos. Esses dados podem ser quantificados e analisados. Os dados de relacionamentos dizem respeito aos contatos, laços e conexões. Esses dados não são dados de um agente, e sim, de um conjunto de agentes conectados que formam um sistema de relacionamentos. É possível quantificar e analisar esses tipos de dados, podendo encontrar padrões de relacionamentos em grupos.

O Software prove mecanismos de análise para melhor planejar e manter a rede de uma organização. Tendo como foco o produto de software e baseando-se nas plataformas tecnológicas, deve-se levar em conta a reutilização de software. Essa necessidade é evidenciada, pois uma empresa geralmente deixa de construir um produto de software isolado, e busca parcerias para construir suas soluções \cite{Lima:2015}.

A reutilização de software é o processo de criar sistemas de software a partir de um software já existente, ao invés de criar a partir do zero \cite{Krueger:1992}.

A meta da reutilização de software é reciclar o \textit{design}, código e outros componentes de um software e, assim, reduzir o custo, o tempo e melhorar a qualidade do produto \cite{Keswani:Joshi:Jatain:2014}.

Mesmo com todos os benefícios mencionados anteriormente, ao lidar com reutilização de software, deve-se planejar e conhecer bem quais são os objetivos esperados a partir dessa prática, como afirmam os autores Keswani, Joshi e Jatain, em sua publicação  \cite{Keswani:Joshi:Jatain:2014}, sobre reutilização de software: ``\textit{Para que um programa de reutilização de software confira o retorno apropriado, o mesmo deve ser sistematizado e planejado. Uma organização que implementa reutilização deve identificar os melhores métodos e estratégias para alcançar máxima produtividade}''.

\section{Questão de Pesquisa}

Esse trabalho tem como intuito colaborar com a seguinte questão: É possível oferecer um \textit{framework} que auxilie no desenvolvimento de redes sociais, disponibilizando recursos gerais de relacionamentos e específicos de definições de trajetos e agenda, proporcionando ao desenvolvedor facilidade ao lidar com preocupações intrínsecas desse contexto?

\section{Justificativa}

As redes sociais ultrapassaram o âmbito acadêmico/científico, conquistando espaço em outras esferas. É possível observar esse movimento em esferas como a Internet, onde encontramos adeptos com objetivos específicos \cite{Tomae:Alcara:Chiara:2005}.

Nas últimas décadas, o trabalho pessoal em redes de conexões passou a ser percebido como um instrumento organizacional, apesar de o envolvimento das pessoas em redes fazer parte da história da humanidade há muito tempo \cite{Tomae:Alcara:Chiara:2005}.

Com tal crescimento e repercussão torna-se, pertinente a criação de um suporte que auxilie a produção de novas redes sociais digitais. Diante dessa questão, este trabalho propõe a criação de um \textit{framework} em atendimento a esse tópico, focando em redes mais específicas, as quais fazem uso de definições de trajetos e rotas ou controle de relacionamento de agendas.

\section{Objetivos}

Esse trabalho contém objetivos geral e específicos, conforme colocado nos subtópicos a seguir apresentados.

\subsection{Objetivo Geral}

Oferecer um \textit{framework} para ser utilizado no desenvolvimento de redes sociais, o qual disponibiliza recursos gerais de relacionamentos e específicos de definição de trajetos e agenda, procurando auxiliar o desenvolvedor de software a lidar com preocupações intrínsecas desse contexto.

\subsection{Objetivos Específicos}

A partir do objetivo geral, pôde-se definir os objetivos específicos. Portanto, têm-se:

\begin{itemize}
	\item Aprofundar os conhecimentos em termos de redes sociais e afins, através da leitura de materiais bibliográficos.
	\item Definir uma arquitetura com base no uso de grafos, alinhada às boas práticas da Engenharia de Software, para representar os relacionamento entre as pessoas.
	\item Usar estruturas de dados e algoritmos específicos, implementados de forma otimizada, visando desempenho e facilidades na manutenção evolutiva do software.
	\item Instanciar um produto de software - i.e. uma rede social específica - a partir do \textit{framework}, no intuito de coletar as primeiras impressões acerca do suporte desenvolvido como tema foco desse trabalho.
	\item Orientar-se por uma metodologia que permita trabalhar a investigação do domínio de redes sociais, a identificação de suporte tecnológico adequado, o desenvolvimento do \textit{framework}, a instanciação do mesmo, e a coleta das primeiras impressões com base nos resultados obtidos.
\end{itemize}

\section{Organização Do Documento}

Este documento foi dividido em seis capítulos que são descritos brevemente à seguir:

\begin{itemize}
	\item \textbf{Referencial Teórico:} Levantamento bibliográfico de tópicos relevantes para alcançar o embasamento teórico necessário para a realização deste trabalho.
	\item \textbf{Suporte Tecnológico:} Exposição das principais ferramentas e tecnologias que serão utilizadas no decorrer deste trabalho.
	\item \textbf{Metodologia:} Apresentação de como será conduzido o trabalho, abordando modelos de desenvolvimento e modelos de pesquisa utilizados.
	\item \textbf{Proposta:} Apresentação completa do que este trabalho se propõe a fazer.
	\item \textbf{Considerações Finais:} Apresentação dos resultados conquistados até o momento bem como de como se dará a continuidade do desenvolvimento.
\end{itemize}
