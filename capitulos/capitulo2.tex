\chapter{Referencial Teórico}

% *****************************************************************
% Redes Sociais, Grafos, Algoritmos para rotas e agenda, Desenvolvimento de Frameworks, Serviços RestFul, Padrões de Projeto
% *****************************************************************

\section{Redes Sociais}

Quando se pensa em rede, surge a ideia de um conjunto de nós interligados entre si, como uma teia que ocupa um determinado espaço em um ambiente. Os nós ou pontos estão ligados em pares e podem representar várias situações em áreas de interesse em comum \cite{Newman:2010}.

Para Sodré \cite{Sodre:2002}, rede é onde as conexões e as interseções tomam o lugar do que seria antes apenas linearidade. Essas conexões e interações ocorrem pelo contato direto, face a face, e pelo contato indireto, utilizando-se um veículo mediador, como o telefone.

As redes sociais constituem uma das estratégias utilizadas pela sociedade para o compartilhamento da informação e do conhecimento, geralmente é formada mediante as relações entre indivíduos submetidos às mesmas pressões sociais ou que enfrentam idênticas dificuldades e obstáculos \cite{Tomae:Alcara:Chiara:2005}.

As pessoas estão inseridas na sociedade por meio das relações que desenvolvem durante toda sua vida, primeiro no âmbito familiar, em seguida na escola, na comunidade em que vivem e no trabalho; enfim, as relações que as pessoas desenvolvem e mantêm é que fortalecem a esfera social. A própria natureza humana nos liga a outras pessoas e estrutura a sociedade em rede \cite{Tomae:Alcara:Chiara:2005}.

Os tipos de relações também podem ser de movimentação entre lugares, como migração, mobilidade física ou social, conexão física, como uma estrada, um rio ou uma ponte que conecta dois lugares, de relações de autoridade ou relação biológica, como descendência, por exemplo \cite{Wasserman:1994}.

Com base em seu dinamismo, as redes, dentro do ambiente organizacional, funcionam como espaços para o compartilhamento de informação e do conhecimento. Espaços que podem ser tanto presenciais quanto virtuais, em que pessoas com os mesmos objetivos trocam experiências, criando bases e gerando informações relevantes para o setor em que atuam \cite{Tomae:Alcara:Chiara:2005}.

\begin{quote}
	``[...] na era da informação – na qual vivemos – as
	funções e processos sociais organizam-se cada vez
	mais em torno de redes. Quer se trate das grandes
	empresas, do mercado financeiro, dos meios de
	comunicação ou das novas ONGs globais,
	constatamos que a organização em rede tornou-se
	um fenômeno social importante e uma fonte crítica
	de poder.'' \cite{Capra:2002}
\end{quote}

O contexto em que estamos inseridos desencadeia uma série de mudanças na rotina dos indivíduos, e uma delas evidencia as redes como ponto de convergência da informação e do conhecimento \cite{Tomae:Alcara:Chiara:2005}.

O conhecimento que a rede possui repercute sobre o meio que esta se encontra, pois Wellman \cite{Wellman:1996} verifica, na rede, sua identidade singular em determinada situação, isto é, a representação e a interpretação das relações em rede estão fortemente ligadas à realidade que a cerca; a rede é influenciada pelo seu contexto e esse por ela.

Portanto os efeitos das redes podem ser percebidos fora de seu espaço, nas interações com o Estado, a sociedade ou outras instituições representativas \cite{Marteleto:2001}.

A interação constante ocasiona mudanças estruturais e, em relação às interações em que a troca é a informação, a mudança estrutural que pode ser percebida é a do conhecimento, quanto mais informação trocamos com o ambiente que nos cerca, com os atores da nossa rede, maior será nossa bagagem de conhecimento, maior será nosso estoque de informação, e é nesse poliedro de significados que inserimos as redes sociais \cite{Tomae:Alcara:Chiara:2005}.

Milgram, em sua tese \cite{Milgram:1967} defende que qualquer pessoa está distantes de qualquer outra pessoa do mundo, a no máximo seis graus de separação. Essa tese ficou conhecida como ``mundo pequeno'' e ``teoria dos seis degraus''. Sua pesquisa demonstra que a rede social constitui importante recurso profissional e pessoal. Estar em contato com pessoas que conheçam uma pessoa-alvo, em razão de um interesse específico, já é um passo além para a conquista de um objetivo.

\section{Grafos}

O desenvolvimento da análise de redes sociais tem como fundamento a teoria dos grafos, ramo da Matemática iniciado por Leonhard Euler, com a sua demonstração matemática de que não era possível atravessar as sete pontes da cidade de Königsberg, passando por elas somente uma vez \cite{Goular:2014}.

\section{Algoritmos}

\section{Frameworks}

Frameworks compartilham técnicas de reutilização em geral e são considerados uma importante parte da cultura de desenvolvimento no mundo da orientação a objetos \cite{Johnson:1997}.

Fayad e Schimidt em seu artigo \cite{Fayad:Schimidt:1997} sobre frameworks de aplicações orientadas a objetos mostram quais são os principais benefícios no uso de frameworks, que são: modularização, reutilização, extensibilidade e inversão de controle.

\begin{itemize}
	\item \textbf{Modularização:} Frameworks encapsulam e interfaceiam algums detalhes de implementação, isso reduz o esforço necessário para entender e manter partes do software existente, pois basta ao desenvolvedor o que lhe é oferecido sem necessariamente entender qual a implementação que existe dentro do framework.

	\item \textbf{Reutilzação:} As interfaces providas por frameworks ajudam também na reutilização através da devinição de componentes genéricos, que podem ser aplicadas em outras aplicações. Dessa forma, soluções comuns para sistemas diferentes podem ser usadas da mesma forma sem a necessidade de recriação e das mesmas. Entende-se então, que estas soluções são pensadas uma única vez e ao estarem presentes em um framework basta que sejam usadas.

	\item \textbf{Extensibilidade:} Esta é um dos principais pontos dos frameworks, fornecem métodos e interfaces estáveis que outras aplicações irão usar, e essas aplicações devem poder usar esses métodos visando resolver problemas parecidos em diferentes contextos. Uma boa estrutura de extensibilidade é essencial para garantir a customização de novos serviços e funcionalidades das aplicações.

	\item \textbf{Inversão de controle:} A inversão de controle ocorre pois a forma como serão processados e entendidos muitos dos eventos de uma aplicação fica invisível ao desenvolvedor quando este usa um framework, pois é o próprio framework quem decide o conjunto de métodos que será invocado para realizar uma determinada tarefa da aplicação.
\end{itemize}

\subsection{Frameworks e Reutilização de Software}

A tecnologia de reutilização ideal provê componentes que podem facilmente ser conectados para criar um novo sistema. Não é necessário ao desenvolvedor ter conhecimento de como o componente é implementado e é fácil para ele aprender como o usar. O resultado é que o sistema será eficiente, fácil de manter e confiável \cite{Johnson:1997}.

Frameworks são aplicações especializadas em prover classes e componentes abstrados que podem ser usados por outros sistemas. Provêem técnicas de reutilização resistentes e de maior granularidade. Sendo aplicações independentes é mais fácil usá-los em um maior número de sistemas \cite{Johnson:Foote:1988}.

Para se alcançar a aplicação efetiva de um dado framework é necessário ao desenvolvedor conhecer as interfaces que o framework proporciona antes de poder usá-las. Como podem existir diversas interfaces complexas aprender a usar um novo framework pode ser difícil. Porém, os frameworks são poderosos e o esforço gasto em sua aprendizagem é recompensado, pois podem reduzir a quantidade de esforço aplicado para se desenvolver uma nova aplicação que os usem \cite{Johnson:1997}.

Ao longo do tempo tornou-se muito caro desenvolver aplicações complexas a partir do zero. Isso porque todos os componentes que são desenvolvidos devem passar por um criterioso processo de validação e manutenção e isso ocorre sempre que um novo sistema é desenvolvido. Ao se usar frameworks pode-se desenvolver componentes comuns e os processos citados são feitos em um único local \cite{Fayad:Schimidt:1997}.

As técnicas de reutilização são diferentes de acordo como o tipo do framework utilizado, esses tipos podem ser "\textit{white box}" ou "\textit{black box}". O primeiro diz respeito a quando o código do framework é aberto e visível ao desenvolvedor, dessa forma, este pode estudar a implementação do framework e modificar o código de determinadas partes de acordo com suas necessidades. Os frameworks do tipo "\textit{black box}" disponibilizam apenas interfaces ao desenvolvedor para que este possa usá-las, a forma como tudo é implementado e processado é desconhecida. No primeiro tipo têm-se uma maior flexibilidade, porém, o uso é mais complexo ao desenvolvedor. No segundo o uso é bem simples, porém, não existe flexibilidade para mudança da implementação \cite{Kroth:2000}.

Além dos tipos de frameworks, estes também podem ser divididos quanto a sua aplicabilidade. Podem ser desenvolvidos para serem aplicados em qualquer domínio, de forma genérica sem se preocupar com algo específico \textit{frameworks horizontais}, ou são desenvolvidos visando atender um tipo específico de domínio de problemas, \textit{frameworks verticais}, essas caracteríscas dependem das necessidades apresentadas ao se trabalhar com frameworks e isso impacta como será aplicada a reutilização \cite{Kroth:2000}.

\section{Padrões de Projeto}

\section{Serviços RestFul}