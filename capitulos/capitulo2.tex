\chapter{Referencial Teórico}

% *****************************************************************
% Redes Sociais, Grafos, Algoritmos para rotas e agenda, Desenvolvimento de Frameworks, Serviços RestFul, Padrões de Projeto
% *****************************************************************

\section{Redes Sociais}

Quando se pensa em rede, surge a ideia de um conjunto de nós interligados entre si, como uma teia que ocupa um determinado espaço em um ambiente. Os nós ou pontos estão ligados em pares e podem representar várias situações em áreas de interesse em comum \cite{Newman:2010}.

Para Sodré \cite{Sodre:2002}, rede é onde as conexões e as interseções tomam o lugar do que seria antes apenas linearidade. Essas conexões e interações ocorrem pelo contato direto, face a face, e pelo contato indireto, utilizando-se um veículo mediador, como o telefone.

As redes sociais constituem uma das estratégias utilizadas pela sociedade para o compartilhamento da informação e do conhecimento, geralmente é formada mediante as relações entre indivíduos submetidos às mesmas pressões sociais ou que enfrentam idênticas dificuldades e obstáculos \cite{Tomae:Alcara:Chiara:2005}.

As pessoas estão inseridas na sociedade por meio das relações que desenvolvem durante toda sua vida, primeiro no âmbito familiar, em seguida na escola, na comunidade em que vivem e no trabalho; enfim, as relações que as pessoas desenvolvem e mantêm é que fortalecem a esfera social. A própria natureza humana nos liga a outras pessoas e estrutura a sociedade em rede \cite{Tomae:Alcara:Chiara:2005}.

Os tipos de relações também podem ser de movimentação entre lugares, como migração, mobilidade física ou social, conexão física, como uma estrada, um rio ou uma ponte que conecta dois lugares, de relações de autoridade ou relação biológica, como descendência, por exemplo \cite{Wasserman:1994}.

Com base em seu dinamismo, as redes, dentro do ambiente organizacional, funcionam como espaços para o compartilhamento de informação e do conhecimento. Espaços que podem ser tanto presenciais quanto virtuais, em que pessoas com os mesmos objetivos trocam experiências, criando bases e gerando informações relevantes para o setor em que atuam \cite{Tomae:Alcara:Chiara:2005}.

\begin{quote}
	``[...] na era da informação – na qual vivemos – as
	funções e processos sociais organizam-se cada vez
	mais em torno de redes. Quer se trate das grandes
	empresas, do mercado financeiro, dos meios de
	comunicação ou das novas ONGs globais,
	constatamos que a organização em rede tornou-se
	um fenômeno social importante e uma fonte crítica
	de poder.'' \cite{Capra:2002}
\end{quote}

O contexto em que estamos inseridos desencadeia uma série de mudanças na rotina dos indivíduos, e uma delas evidencia as redes como ponto de convergência da informação e do conhecimento \cite{Tomae:Alcara:Chiara:2005}.

O conhecimento que a rede possui repercute sobre o meio que esta se encontra, pois Wellman \cite{Wellman:1996} verifica, na rede, sua identidade singular em determinada situação, isto é, a representação e a interpretação das relações em rede estão fortemente ligadas à realidade que a cerca; a rede é influenciada pelo seu contexto e esse por ela.

Portanto os efeitos das redes podem ser percebidos fora de seu espaço, nas interações com o Estado, a sociedade ou outras instituições representativas \cite{Marteleto:2001}.

A interação constante ocasiona mudanças estruturais e, em relação às interações em que a troca é a informação, a mudança estrutural que pode ser percebida é a do conhecimento, quanto mais informação trocamos com o ambiente que nos cerca, com os atores da nossa rede, maior será nossa bagagem de conhecimento, maior será nosso estoque de informação, e é nesse poliedro de significados que inserimos as redes sociais \cite{Tomae:Alcara:Chiara:2005}.

Milgram, em sua tese \cite{Milgram:1967} defende que qualquer pessoa está distantes de qualquer outra pessoa do mundo, a no máximo seis graus de separação. Essa tese ficou conhecida como ``mundo pequeno'' e ``teoria dos seis degraus''. Sua pesquisa demonstra que a rede social constitui importante recurso profissional e pessoal. Estar em contato com pessoas que conheçam uma pessoa-alvo, em razão de um interesse específico, já é um passo além para a conquista de um objetivo.

\section{Grafos}

O desenvolvimento da análise de redes sociais tem como fundamento a teoria dos grafos, ramo da Matemática iniciado por Leonhard Euler, com a sua demonstração matemática de que não era possível atravessar as sete pontes da cidade de Königsberg, passando por elas somente uma vez \cite{Goular:2014}.

\section{Algoritmos}

\section{Frameworks}

\section{Padrões de Projeto}

\section{Serviços RestFul}