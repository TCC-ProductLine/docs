\chapter{Conclusão}
\label{chapter:Conclusao}
Este trabalho teve como objetivo o desenvolvimento do SocialFramework, que é um \textit{framework} para desenvolvimento de redes sociais, porém, é focado em um nicho específico de rotas e agendas.

No decorrer do desenvolvimento da aplicação, foram utilizados padrões de projetos, os princípios das técnicas de programação, desenvolvimento de testes, monitoramento da análise estática do código, além de outros princípios da engenharia de software. A intenção foi procurar obter um \textit{framework} orientado por critérios de qualidade, no caso, facilidade de manutenção e evolução de software.

Neste trabalho, foi levantada a seguinte questão de pesquisa: ``É possível oferecer um \textit{framework} que auxilie no desenvolvimento de redes sociais, disponibilizando recursos gerais de relacionamentos e específicos de definições de rotas e agenda, proporcionando ao desenvolvedor facilidade ao lidar com preocupações intrínsecas desse contexto?'' Com base nos resultados apresentados e na instanciação de uma rede social que possuía como base o \textit{framework} desenvolvido, pode-se concluir que a resposta para essa questão é: Sim, o SocialFramework atingiu os objetivos estabelecidos.

Verificou-se que todos os objetivos esperados foram atingidos com o desenvolvimento do SocialFramework e da rede social SocialBike. A seguir, são apresentadas algumas sugestões para trabalhos futuros, em especial, aqueles que partem dos incrementos que foram desenvolvidos até o momento.

\section{Trabalhos Futuros}

A seguir, serão apresentados os principais pontos que deverão ser evoluídos para a continuidade do \textit{framework}.

\begin{enumerate}
	\item Implementar, no módulo de usuários, suporte para grupos;
	\item Desenvolver suporte para postagens de usuários;
	\item Inserir, no módulo de usuários, suporte para \textit{chat};
	\item Evoluir o módulo de agenda para oferecer suporte à repetição de eventos;
	\item Integração da agenda do \textit{framework} com a agenda do Google.
\end{enumerate}
