\chapter{Considerações Finais}
\label{chapter:Consideracoes_Finais}

 Este documento completa a primeira etapa deste trabalho, onde foram apresentados todos os principais pontos que comtemplam a visão geral do tema proposto. Esta visão, em um contexto mais amplo, compreende aspectos relacionados a redes sociais e reutilização de software e apresenta a ideia geral de desenvolvimento de um \textit{framework}. Esse suporte auxiliará na criação de redes sociais, oferecendo recursos gerais de relacionamento entre pessoas em uma rede virtual, bem como recursos específicos de rotas e agendas. A partir desses recursos, um desenvolvedor poderá usar este \textit{framework} usufruindo do suporte sem se preocupar com a lógica por trás de sua implementação.

 Em um contexto mais técnico, foram discutidas também algumas formas para desenvolver o que se propõe. Entre esses aspectos, foram apresentados pontos como teoria dos grafos, algoritmos relacionados e padrões de projeto.

Durante a contrução deste documento, alguns passos foram seguidos buscando apresentar o tema e aspectos relacionados. Inicialmente, foi desenvolvida uma contextualização que traz o tema proposto de uma forma geral e apresenta ao leitor os objetivos deste trabalho. Logo após, foi feito um levantamento bibliográfico com o objetivo de buscar uma melhor adequação ao tema proposto. Esse passo colaborou com a elaboração do capítulo de ~\nameref{chapter:Referencial_Teorico}, onde estão presentes todos os temas pertinentes a esse trabalho, levantados até o momento. Foi feito também um levantamento das tecnologias e ferramentas que serão usadas durante este trabalho, as quais estão documentadas no capítulo de ~\nameref{chapter:Suporte_Tecnologico}. No capítulo de ~\nameref{chapter:Metodologia}, foram definida as metodologias de pesquisa e de desenvolvimento. Por fim, concretizou-se a ~\nameref{chapter:Proposta} deste trabalho apresentando-a de uma forma mais completa e discorrendo sobre os principais pontos de forma mais técnica. Nessa etapa, também foi desenvolvida uma aplicação que serviu como prova de conceito para aprofundar os conhecimentos quanto às necessidades e dificuldades relacionadas à proposta, ou seja, ao desenvolvimento do \textit{framework} bem como de duas instanciações: uma voltada para o domínio de rotas e outra voltada para o domínio de agendas.

Com isso, a primeira etapa desta trabalho está completa. A próxima etapa levará em consideração tudo que foi investigado, desenvolvido e contemplado até o momento. Posteriormente, serão inicializadas as etapas que compreendem maior esforço de desenvolvimento. Na primeira etapa de desenvolvimento, os requisitos detalhados do \textit{framework} serão levantados, o que conduzirá à especificação do \textit{backlog} do produto. Esse, por sua vez, servirá de base para a criação e a priorização das \textit{sprints}, onde de fato o código do \textit{framework} será desenvolvido. A segunda etapa compreende o desenvolvimento de uma aplicação que instanciará o \textit{framework} proposto. Por fim, os resultados serão coletados e apresentados para finalização de todo o trabalho.
