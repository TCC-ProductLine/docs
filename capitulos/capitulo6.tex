\chapter{Conclusão}
\label{chapter:Conclusao}
Este trabalho teve como objetivo o desenvolvimento do SocialFramework, que é um \textit{framework} para desenvolvimento de redes sociais, porém é focado em um nicho específico de rotas e agendas.

No decorrer do desenvolvimento da aplicação, foram utilizados padrões de projetos, os princípios das técnicas de programação, desenvolvimento de testes, monitoramento da análise estática do código, além de outros princípios da engenharia de software, visando um \textit{framework} de qualidade, para que a sua manuteção e evolução e reutilização se dê de uma maneira simple.

Neste trabalho, foi levantada a seguinte questão de pesquisa: ``É possível oferecer um framework que auxilie no desenvolvimento de redes sociais, disponibilizando recursos gerais de relacionamentos e específicos de definições de rotas e agenda, proporcionando ao desenvolvedor facilidade ao lidar com preocupações intrínsecas desse contexto?'' Com base nos resultados apresentados e na instânciação de uma rede social que possuia como base o \textit{framework} desenvolvido, pode-se concluir que a resposta para essa questão é: Sim, o SocialFramework atingiu os objetivos estabelecidos.

\section{Sugestão de trabalhos futuros}

A seguir será apresentado os principais pontos que deverão ser evoluidos para a continuidade do \textit{framework}.

\begin{enumerate}
	\item Implementar no módulo de usuários suporte para grupos;
	\item Desenvolber suporte para postagens de usuários;
	\item Inserir no módulo de usuários suporte para \textit{chat};
	\item Evoluir o módulo de agenda para oferecer suporte a repetição de eventos;
	\item Integração da agenda do \textit{framework} com a agendo do google;
\end{enumerate}
