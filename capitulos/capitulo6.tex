\chapter{Considerações Finais}
\label{chapter:Consideracoes_Finais}

 Este documento completa a primeira etapa deste trabalho, onde foram apresentados todos os principais pontos que apresentam a visão geral do tema proposto. Esta visão em um contexto mais amplo compreende aspectos relacionados a redes sociais e reutilização de software e apresenta a ideia geral de desenvolvimento de um \textit{framework} que venha auxiliar na criação de redes sociais, oferecendo recursos gerais de relacionamento entre pessoas em uma rede virtual, agenda e rotas, a partir desses recursos um desenvolvedor poderá usar este \textit{framework} e poderá usufruir dos recursos providos pelo mesmo sem se preocupar com a lógica por trás de sua implementação. Em um contexto mais técnico foram discutidas também algumas formas para desenvolver o que se propõe, entre esses aspectos foram apresentados pontos como teoria dos grafos, algoritmos relacionados e padrões de projeto.

Durante a contrução deste documento algums passos foram seguidos buscando apresentar o tema e aspectos relacionados. Inicialmente foi desenvolvida uma contextualização que traz o tema proposto de uma forma geral e apresenta ao leitor os objetivos deste trabalho. Logo após, foi feito um levantamento bibliográfico com o objetivo de buscar uma melhor adequação ao tema proposto, este passo construiu o capítulo de ~\nameref{chapter:Referencial_Teorico} onde estão presentes todos os temas pertinentes a esse trabalho levantados até o presente momento. Foi feito também um levantamento das tecnologias e ferramentas que serão usadas durante este trabalho e foi definida as metodologias de pesquisa e de desenvolvimento. Por fim, concretizou-se a proposta deste trabalho apresentando-a de uma forma mais completa e discorrendo sobre os principais pontos de uma forma mais técnica, nessa etapa também foi desenvolvida uma aplicação que serviu como prova de conceito para os conceitos que serão usados na etapa de desenvolvimento.

Com isso a primeira etapa desta trabalho está completa. A próxima etapa levará em consideração tudo que foi levantado e a partir disso se dará inicio a etapa de desenvolvimento, onde serão elaborados requisitos detalhados do \textit{framework} que levarão a construção do \textit{backlog} do produto que servirá de base para criação e priorização das \textit{sprints} onde de fato o código do \textit{framework} será desenvolvido. A segunda etapa compreende também o desenvolvimento de uma aplicação que comprovará o uso do \textit{framework}. Por fim, os resultados serão coletados e apresentados para finalização de todo o trabalho.
