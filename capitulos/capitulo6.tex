\chapter{Resultados}
\label{chapter:Resultados}

\section{O framework}

O SocialFramework é uma \textit{engine} do Rails que ajuda a construir redes sociais com recursos comuns e específicos.

O SocialFramework é dividido em três módulos, que são: usuários, rotas e agendas. No módulo de usuários são providos os principais recursos para usuários, como autenticação, registro e pesquisas. No módulo de rotas o \textit{framework} provê recursos para checar a compatibilidade de rotas. No módulo de agenda são oferecidos recursos para definir agendas de usuários e a compatibilidade de horário entre usuários.

Portanto, o SocialFramework pode ajudar a construir redes sóciais genéricas e específicas de um modo rápido e prático e sem se preocupar com problemas recorrentes nestes tipos de sistemas.

\subsection{Instalação}

A instalação do SocialFramework possui dois passos simples.

Primeiro adicione a seguinte linha no arquivo Gemfile:

\begin{lstlisting}[
    label=listing:Gemfile,
    caption=Gemfile,
    numbers=none,
    language=Ruby,
    basicstyle=\footnotesize\sffamily,
    keywordstyle=\color{red},
    stringstyle=\color{blue},
]
gem 'social_framework'
\end{lstlisting}

Após a adição da \textit{gem} no Gemfile, instale-o executando o seguinte comando:

\begin{lstlisting}[
    label=listing:Install,
    caption=Instalação,
    numbers=none,
    language=Bash,
    basicstyle=\footnotesize\sffamily,
    keywordstyle=\color{red},
    stringstyle=\color{blue},
]
$ bundle install
\end{lstlisting}

Isto irá instalar o SocialFramework na aplicação.

\subsection{Primeiros passos}

O SocialFramework utiliza o Devise, que é uma solução de autenticação flexível para Rails. Para visualizar a documentação completa do Devise acesse: \url{https://github.com/plataformatec/devise}. A classe de usuário foi implementada no SocialFramework e possui algumas diferenças da classe de usuário padrão do Devise, como a adição do atributo nome e dos relacionamentos dos usuários. As \textit{controllers} e \textit{views} do Devise também foram alteradas para adicionar os novos recursos.

Inicialmente, alguns arquivos devem ser adicionado na aplicação para que configurações do SocialFramework e do Devise sejam adicionadas. Esses arquivos são os \textit{initializers}, o arquivo de internacionalização (i18n) do Devise, as rotas e as \textit{views} de \textit{registration} e \textit{session} para criar e autenticar usuários. Para isso deve-se executar:

\begin{lstlisting}[
    label=listing:generate_social_framework,
    caption=Gerando as configurações do SocialFramework e do Devise,
    numbers=none,
    language=Bash,
    basicstyle=\footnotesize\sffamily,
    keywordstyle=\color{red},
    stringstyle=\color{blue},
]
$ rails generate social_framework:install
\end{lstlisting}

Este comando irá criar o arquivo ``config/initializers/devise.rb'' contendo as configurações do Devise, o arquivo ``config/initializers/social\_framework.rb'' que contem as configurações do SocialFramework, o arquivo i18n do Devise e irá adicionar a rota ``devise\_for'' para mapear as \textit{controllers} e \textit{views} do Devise. Com isto a aplicação está preparada para usuar o módulo de usuários com as configurações e comportamentos padrões.

Para testar a aplicação não se deve esquecer de executar as migrations:

\begin{lstlisting}[
    label=listing:migrations,
    caption=Migrations,
    numbers=none,
    language=Bash,
    basicstyle=\footnotesize\sffamily,
    keywordstyle=\color{red},
    stringstyle=\color{blue},
]
$ rake db:create
$ rake db:migrate
\end{lstlisting}

Todas as tabelas do \textit{framework} serão criadas no banco de dados da aplicação.

Para acessar a página de autenticação acesse a rota ``/users/sing\_in'', esta página está preparada para autenticar usuários com \textit{email} ou nome de usuário. Para cadastrar um novo usuário deve-se acessar a rota ``/users/sing\_up'', ao criar um novo usuário você estará automaticamente conectado.

Quando se usa o Devise Mailer como o Módulo confirmável é necessário adicionar em seu ambiente as configurações para a ação \textit{mailer}, por exemplo, se estiver no ambiente de desenvolvimento deve-se adicionar as seguintes configurações no arquivo `development.rb'.

\begin{lstlisting}[
    label=listing:mailer,
    caption=Configurações de email,
    numbers=left,
    language=Ruby,
    basicstyle=\footnotesize\sffamily,
    keywordstyle=\color{red},
    stringstyle=\color{blue},
    showspaces=false,
    showstringspaces=false,
]
config.action_mailer.default_url_options = {host: 'localhost', port: 3000}

config.action_mailer.delivery_method = :smtp

config.action_mailer.smtp_settings = {
  address: "smtp.gmail.com",
  port: 587,
  domain: ENV["GMAIL_DOMAIN"],
  authentication: "plain",
  enable_starttls_auto: true,
  user_name: ENV["GMAIL_USERNAME"],
  password: ENV["GMAIL_PASSWORD"]
}
\end{lstlisting}

Pode-se alterar os valores em `domain', `user\_name', e `password' ou criar as variáveis de ambiente locais, isso é indicado para esconder essas informações sigilosas e garantir uma maior segurança. As mesmas configurações são válidas para os ambientes de teste e produção, nos arquivos `test.rb' e `production.rb'.

\section{Hotspots}

\section{Relatório de testes}

\section{Relatório de qualidade}

\section{Relatório de desempenho}

% \subsecton{Tempo de execução}

% \subsecton{Memória}

\section{Uso da rede social}
