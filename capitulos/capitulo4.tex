\chapter{Metodologia}

As metodologias específicas são necessárias para ajudar a estabelecer uma base de engenharia e de ciência para a Engenharia de Software \cite{Wohlin:2000}.

A metodologia pode ser entendida como etapas que devem ser realizadas quando se irá realizar investigações a respeito de um determinado assunto. É a metodologia que define os passos que serão seguidos para realização das pesquisas, como escolha do tema, planejamento da investigação, desenvolvimento, coleta e análise dos dados, análise dos resultados e conclusões a respeito \cite{Moresi:2003}.

Pesquisar significa identificar uma dúvida que necessite ser esclarecida, construir e executar o processo que apresenta a solução desta, quando não há teorias que a expliquem ou quando as teorias que existem não estão aptas para fazê-lo \cite{Koche:1997}. A seguir será apresentado as formas de se classificar uma pesquisa.

\section{Classificação da pesquisa}

\begin{itemize}
	\item Do ponto de vista da natureza da pesquisa esta pode ser:
		\begin{itemize}
			\item \textbf{Pesquisa Básica:} Possui o objetivo de gerar novos conhecimentos para a ciência, não é obrigatário que este conhecimento gere um uso prático \cite{Silva:Tafner:2007}.
			\item \textbf{Pesquisa Avançada:} Visa gerar uma maior compreensão para assuntos práticos dirigido à solução de problemas específicos \cite{Silva:Tafner:2007}.
		\end{itemize}

	\item Do ponto de vista da forma de abordagem do problema pode ser:
		\begin{itemize}
			\item \textbf{Pesquisa Quantitativa:} O estudo quantitativo considera que tudo pode ser quantificável, ou seja, que os números podem ser classificados, gerando informações ao analisá-los, através da análise estatística \cite{Travassos:2002}.
			\item \textbf{Pesquisa Qualitativa:} O estudo qualitativo está relacionado à pesquisa sobre os objetos quando os resultados são apresentados em termos naturais \cite{Travassos:2002}.
		\end{itemize}

	\item Do ponto de vista de seus objetivos pode ser:
		\begin{itemize}
			\item \textbf{Pesquisa Exploratória:} Esta pesquisa têm como objetivo proporcionar maior familiaridade com o problema, possibilitando o aprimoramento de ideias ou a descoberta de intuições \cite{Gil:2010}.
			\item \textbf{Pesquisa Descritiva:} Seu principal objetivo é a descrição das características de determinada população ou fenômeno ou, então, o estabelecimento de relações entre variáveis \cite{Gil:2010}.
			\item \textbf{Pesquisa Explicativa:} Têm como preocupação central identificar os fatores que determinam ou que contribuem para a ocorrência dos fenômenos. Esse é o tipo de pesquisa que mais aprofunda o conhecimento da realidade, porque explica a razão, o porquê das coisas \cite{Gil:2010}.
		\end{itemize}

	\item Do ponto de vista dos procedimentos técnicos pode ser:
		\begin{itemize}
			\item \textbf{Pesquisa Bibliográfica:} Visa encontrar as fontes primárias e secundárias e os materiais científicos e tecnológicos necessários para a realização do trabalho científico ou técnico-científico. Muitos os estudos fazem uso do levantamento bibliográfico ou são desenvolvidas exclusivamente por fontes bibliográficas. Sua principal vantagem é possibilitar ao investigador a cobertura de uma gama de acontecimentos muito ampla \cite{Silva:Tafner:2007}.

			\item \textbf{Pesquisa Documental:} Se assemelha a pesquisa bibliográfica. Porém, esta é realizada partir de materiais que não receberam tratamento analítico. Por exemplo, reportagens de jornal, cartas, contratos, diários, filmes, fotografias e gravações. Ou ainda documentos de segunda mão, que de alguma forma já foram analisados. Por exemplo, relatórios de empresas e tabelas estatísticas \cite{Gil:2010}.

			\item \textbf{Levantamento:} É uma investigação realizada em retrospecto, que em seguida, mediante análise quantitativa, chega as conclusões correspondentes aos dados coletados. O levantamento feito com informações de todos os integrantes do universo da pesquisa origina um censo. \cite{Mafra:Travassos:2006}.
			
			\item \textbf{Estudo de Caso:} São estudos conduzidos com o propósito de se investigar uma entidade ou um fenômeno dentro de um espaço de tempo específico. Estes são usados principalmente para a monitoração de atributos presentes em projetos, atividades ou atribuições. Durante a sua condução, dados são coletados e analisados estatisticamente de forma, a permitir a avaliação de um determinado atributo ou o relacionamento entre diferentes atributos. \cite{Mafra:Travassos:2006}
			
			\item \textbf{Pesquisa-Ação:} É realizada em conjunto com uma ação ou com a resolução de um problema coletivo. Visando definir o campo de investigação, as expectativas dos interessados e o tipo de auxílio que estes poderão exercer ao longo do processo de pesquisa. Esta pesquisa, implica no contato direto com o campo de estudo, envolvendo o reconhecimento visual do local, consulta a documentos diversos e a discussão com os envolvidos na pesquisa. A abordagem dos problemas dos grupos investigados na pesquisa-ação é mais qualitativo do que quantitativo \cite{Silva:Tafner:2007}.
			
			\item \textbf{Pesquisa Participante:} A intenção é obter um maior conhecimento sobre o grupo. O grupo investigado tem ciência da finalidade, dos objetivos da pesquisa e da identidade do pesquisador. A pesquisa participante, permite a observação das ações no próprio momento em que ocorrem \cite{Silva:Tafner:2007}.
			
			\item \textbf{Pesquisa Experimental:} São conduzidos quando deseja-se obter um maior controle da situação, ao manipular-se as variáveis envolvidas no estudo de forma direta, sistemática e precisa. A pesquisa experimental necessita de previsão de relações entre as variáveis a serem estudadas, como também o seu controle. O objetivo é manipular uma ou mais variáveis e controlar todas as outras variáveis num valor fixo. O efeito da manipulação das variáveis é então medido e, baseado nessa medição, análises estatísticas são conduzidas. A condução de experimentos reais é rara em Engenharia de Software, devido à dificuldade de se alocar os participantes do estudo a diferentes tratamentos de forma aleatória. Nessas situações, tais estudos denominam-se quasi-experimentos \cite{Mafra:Travassos:2006}.
			
			\item \textbf{Pesquisa Ex-Post-Facto:} Realizada quando o experimento se acontece depois dos fatos. Neste caso, o pesquisador não tem controle sobre as variáveis. Esta pesquisa difere da da pesquisa experimental pelo fato de o fenômeno ocorrer naturalmente sem que o pesquisador tenha controle sobre ele, ou seja, o pesquisador passa a ser um mero observador do acontecimento \cite{Silva:Tafner:2007}.
		\end{itemize}

\end{itemize}

\section{Planejamento da Metodologia Aplicada}

Este trabalho possui metodologias específicas para a realização do levantamento bibliográfico e do desenvolvimento das aplicações propostas.

A seguir serão apresentadas as duas metodologias e como estas serão aplicadas ao logo deste trabalho.

\subsection{Metodologia de Pesquisa}

Analisando-se o tema proposto para este trabalho pode-se ver que a pesquisa caracteriza uma pesquisa avançada, pois, todo levantamento bibliográfico visa a geração de compreensões a cerca dos temas e estes conhecimentos adquiridos deverão ser aplicados no desenvolvimento. A abordagem é tanto quantitativa quanto qualitativa.

Com relação aos objetivos da pesquisa, considera-se que é uma pesquisa exploratória, devido a necessidade encontrada em entender melhor o problema apresentado e se obter um melhor entendimento de todos os temas relacionados ao que se propõe.

Por fim, no contexto dos procedimentos técnicos será aplicada a pesquisa bibliográfica visando o levantamento de uma ampla gama de fontes para se alcançar o conhecimento necessário para a realização do trabalho. Além da pesquisa exploratória têm-se também a aplicação da pesquisa-ação, que deverá ocorrer no decorrer do desenvolvimento caso haja falta de algum ponto que não tenha sido coberto pela pesquisa bibliográfica.

\subsection{Metodologia de Desenvolvimento}

A metodologia de desenvolvimento seguirá algumas práticas ágeis já consolidadas. A organização do trabalho seguirá práticas do \textit{Scrum}\footnote{\url{http://www.desenvolvimentoagil.com.br/scrum/}} com o uso de estórias de usuário, reuniões diárias, \textit{sprints} de quinze dias, \textit{backlog do produto} e \textit{backlog de sprint}. Serão aplicadas também algumas práticas do \textit{XP}\footnote{\url{http://www.desenvolvimentoagil.com.br/xp/}} como o uso de integração contínua, pareamento e \textit{planning poker} para estimativa de pontos das estórias.

\subsection{Fluxo de trabaho}
\subsection{Cronograma}
