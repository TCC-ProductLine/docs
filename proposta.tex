\documentclass[12pt,openright,twoside,a4paper,english,french,spanish]{abntex2}

\usepackage[alf]{abntex2cite}
\usepackage[brazilian,hyperpageref]{backref}
\usepackage[T1]{fontenc}
\usepackage[utf8]{inputenc}
\usepackage{bold-extra}
\usepackage{cmap}
\usepackage{color}
\usepackage{eso-pic}
\usepackage{graphicx}
\usepackage{indentfirst}
\usepackage{lastpage}
\usepackage{lmodern}
\usepackage{units}
\usepackage{booktabs}
\usepackage{siunitx}
\usepackage{graphicx}
\usepackage{wrapfig}
\input{fixos/comandos}
\input{fixos/novosComandos}

% Dados pessoais
\autor{Álex Silva Mesquita, Jefferson Nunes de Sousa Xavier}
\curso{Engenharia de Software}

% Dados do trabalho
\titulo{Framework para Redes Sociais}
\data{2015}
\palavraChaveUm{Palavra-chave 01}
\palavraChaveDois{Palavra-chave 02}

% Dados da orientacao
\orientador{Prof. Dr. Maurício Serrano}
\coorientador{Profª. Drª. Milene Serrano}

% Dados para a ficha catalográfica
\cdu{02:141:005.6}

% Dados da aprovação do trabalho
\dataDaAprovacao{01 de junho de 2015}
\membroConvidadoUm{Titulação e Nome do Professor Convidado 01}
\membroConvidadoDois{Titulação e Nome do Professor Convidado 02}

% Dados pessoais
\autor{Álex Silva Mesquita, Jefferson Nunes de Sousa Xavier}
\curso{Engenharia de Software}

% Dados do trabalho
\titulo{Framework para Redes Sociais}
\data{2015}
\palavraChaveUm{Palavra-chave 01}
\palavraChaveDois{Palavra-chave 02}

% Dados da orientacao
\orientador{Prof. Dr. Maurício Serrano}
\coorientador{Profª. Drª. Milene Serrano}

% Dados para a ficha catalográfica
\cdu{02:141:005.6}

% Dados da aprovação do trabalho
\dataDaAprovacao{01 de junho de 2015}
\membroConvidadoUm{Titulação e Nome do Professor Convidado 01}
\membroConvidadoDois{Titulação e Nome do Professor Convidado 02}

\definecolor{blue}{RGB}{41,5,195}
\makeatletter
\hypersetup{
     	%pagebackref=true,
		pdftitle={\@title}, 
		pdfauthor={\@author},
    	pdfsubject={\imprimirpreambulo},
	    pdfcreator={LaTeX with abnTeX2},
		pdfkeywords={abnt}{latex}{abntex}{abntex2}{trabalho acadêmico}, 
		colorlinks=true,       		% false: boxed links; true: colored links
    	linkcolor=blue,          	% color of internal links
    	citecolor=blue,        		% color of links to bibliography
    	filecolor=magenta,      		% color of file links
		urlcolor=blue,
		bookmarksdepth=4
}
\makeatother
\setlength{\parindent}{1.3cm}
\setlength{\parskip}{0.2cm}  
\makeindex

\definecolor{codegreen}{rgb}{0,0.6,0}
\definecolor{codegray}{rgb}{0.5,0.5,0.5}
\definecolor{codepurple}{rgb}{0.58,0,0.82}
\definecolor{backcolour}{rgb}{0.95,0.95,0.92}
\renewcommand\lstlistlistingname{Códigos}
 
\lstdefinestyle{mystyle}{
    backgroundcolor=\color{backcolour},   
    commentstyle=\color{codegreen},
    keywordstyle=\color{magenta},
    numberstyle=\tiny\color{codegray},
    stringstyle=\color{codepurple},
    basicstyle=\footnotesize,
    breakatwhitespace=false,         
    breaklines=true,                 
    captionpos=b,                    
    keepspaces=true,                 
    numbers=left,                    
    numbersep=5pt,                  
    showspaces=false,                
    showstringspaces=false,
    showtabs=false,                  
    tabsize=2
}
 
\lstset{style=mystyle}


\begin{document}

\frenchspacing 
\imprimircapa

\textual

\section*{Contextualização}

Uma rede social virtual é uma comunidade virtual que representa um conjunto de participantes (pessoas, organizações ou outras entidades), unindo ideias e recursos em torno de relacionamentos, valores e interesses compartilhados. [3]

Seguindo-se esse contexto, observa-se diversas características que são comuns a maioria das Redes Sociais. Visando esses aspectos apresenta-se uma solução que possa facilitar o desenvolvimento de sistemas desse tipo, buscando agilidade, segurança e desempenho, e evitando retrabalhos e duplicações de código. As linhas de produto de software podem trazer esses quesitos e é nesse contexto que se segue esse trabalho.

Linha de produto de software é um portfólio de sistemas baseados em software similares e produtos produzidos a partir de um conjunto compartilhado de ativos de software usando um meio comum de produção. [1]

\section*{Problema de Pesquisa}

É possível desenvolver um framework que ofereça recursos comuns para o desenvolvimento de Redes Sociais voltadas para rotas e combinações de horários, evitando assim um retrabalho no desenvolvimento?

\section*{Justificativa}

As linhas de produtos podem ajudar as organizações a superar os problemas causados pela escassez de recursos. Organizações de todos os tipos e tamanhos descobriram que uma estratégia de linha de produtos, quando habilmente executada, pode produzir muitos benefícios e, finalmente, dar às organizações uma vantagem competitiva. [2]

\section*{Objetivos}

\begin{enumerate}
	\item \textbf{Objetivo Geral}: Desenvolver um framework para ser utilizado no desenvolvimento de Redes Sociais, visando a criação de uma linha de produtos de software.

	\item \textbf{Objetivos Específicos}
	\begin{enumerate}
		\item Definir arquitetura da linha de produtos.
		\item Desenvolver o framework da linha de produtos.
		\item Desenvolver um software para comprovação da eficácia do framework desenvolvido.
		\item Fazer o uso de estruturas de dados e algoritmos mais indicados, visando o melhor custo para desenvolvimento e futuras manutenções e perfomance.
	\end{enumerate}
\end{enumerate}
\section*{Metodologia}

\section*{Cronograma}

\postextual

\bibliography{bibliografia} 

\end{document}

