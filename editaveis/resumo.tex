\begin{resumo}

Redes sociais vem ganhando um grande destaque em meio a sociedade nos tempos atuais. A cada dia novas redes são desenvolvidas e ganham a atenção de diversos usuários no mundo todo. Torna-se necessário o desenvolvimento de alguma ferramenta que auxilie desenvolvedores a criar novas redes de uma maneira mais rápida e prática e sem se preocupar com problemas recorrentes nesse tipo de sistema. Com essa ideia em mente este trabalho se propõe a desenvolver um \textit{framework} que possa fornecer recursos comuns no desenvolvimento de redes sociais, dessa forma, cria-se toda lógica de relacionamento de usuários em uma rede social virtual e essa lógica pode ser extendida e usada em aplicações que venham fazer uso desse \textit{framework}, além desse recurso o \textit{framework} implementará também alguns recursos extras, esses são rcursos de agenda e rotas, sendo assim, um desenvolvedor que use o \textit{framework} apresentado terá em suas mãos esses três recursos citados e poderá aplicar conforme as suas necessidades em uma rede social sem se preocupar com a lógica implementada. Esse aspecto do \textit{framework} com recursos além dos gerais propõe atender redes sociais que tem se voltado a um nicho mais específico de usuários e vem ganhando grande espaço atualmente.

 \vspace{\onelineskip}
    
 \noindent
 \textbf{Palavras-chaves}: \textit{Framework}. Reutilização de software. Redes sociais. Grafos.
\end{resumo}
