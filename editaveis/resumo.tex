\begin{resumo}

Redes sociais vem ganhando um grande destaque em meio a sociedade nos tempos atuais. A cada dia, novas redes são desenvolvidas, merecendo a atenção de diversos usuários no mundo todo. Torna-se necessário o desenvolvimento de alguma ferramenta que auxilie desenvolvedores a criar novas redes de uma maneira mais rápida e prática, sem se preocupar com problemas recorrentes nesse tipo de sistema. No intuito de colaborar com esse contexto tecnológico, este trabalho se propõe a desenvolver um \textit{framework} que possa fornecer recursos de rotas e agendas, bem como recursos comuns no desenvolvimento de redes sociais como, por exemplo, o relacionamento de usuários. Dessa forma, a presente proposta procurará criar a lógica para atender o gerenciamento de rotas e agendas e de relacionamento entre usuários em uma rede social virtual. Pretende-se ainda que essa lógica seja estendida e usada em aplicações que instanciem o suporte proposto, no caso, um \textit{framework}. Dessa forma, um desenvolvedor que instancie o \textit{framework} proposto, terá disponível esses recursos, permitindo aplicar conforme as suas necessidades em uma rede social, sem, necessariamente, se preocupar com a lógica implementada. Tal suporte, o qual transcende recursos mais básicos de redes sociais, procurará atender perfis de usuários, interessados em redes sociais mais específicas. Esse cenário tem se tornado algo desejado nos tempos atuais.

 \vspace{\onelineskip}
    
 \noindent
 \textbf{Palavras-chaves}: \textit{Framework}. Reutilização de software. Redes sociais. Grafos.
\end{resumo}
